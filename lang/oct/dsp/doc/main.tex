\documentclass[12pt, a4paper]{article}

% import packages
\usepackage{MinionPro}
\usepackage{listings}
\usepackage{CJKutf8}
\usepackage{indentfirst}

% configurations
\lstset{%
language=Matlab,%
frame=single,%
numbers=left,%
stringstyle=\ttfamily,%
keepspaces=true,%
tabsize=4,%
basicstyle=\ttfamily,%
}

% titlepage
\begin{titlepage}
\title{数字信号处理大作业}
\author{Zhou Mo}
\date{\today}
\end{titlepage}

% document
\begin{document}
\begin{CJK}{UTF8}{gkai}

\maketitle

在本报告中,DFT和FFT分别被用Matlab语言实现。
为探究FFT相对于DFT的性能提升,一个长度为1024的随机序列
被用于测试,发现FFT相对DFT快$58.1$倍。

\begin{lstlisting}[caption={程序运行结果}]
dft ...
t0 =  22.038
fft ...
t1 =  0.37915
fft reference
t2 =  0.0045090

myfft to dft acceleration ratio 58.125119
fft to dft acceleration ratio 4887.593486
fft is 84.087458 times faster than myfft
\end{lstlisting}

\section{Discrete Fourier Transform}
离散Fourier变换及其逆变换如下:
\begin{align}
X(k) &= \sum_{n=0}^{N-1} x(n)W_N^{kn}, 0 \leq k \leq N-1 \\
x(n) &= \frac{1}{N} \sum_{k=0}^{N-1} X(k)W_N^{-kn}, 0 \leq n \leq N-1
\end{align}
其中\[ W = e^{-j \frac{2\pi}{N}}\]

DFT及IDFT的实现如下。其中,DFT和IDFT均使用迭代算法实现。
\begin{lstlisting}[caption={DFT Program}]
function y = dft (x)

% argument
len = length(x);

% dft
ret = zeros(1, len);
for k = 1:len
    xk = 0;
    for n = 1:len
        xk = xk + x(n) * ...
            exp(j * (k-1) * (n-1) * (2 * pi)/ len);
    end
    ret(k) = xk;
end
y = ret;

end
\end{lstlisting}

\begin{lstlisting}[caption={IDFT Program}]
function y = idft (x)

% argument
len = length(x);

ret = zeros(1, len);
for k = 1:len
    xk = 0;
    for n = 1:len
        xk = xk + x(n) * ...
            exp(-j * (k-1) * (n-1) * (2 * pi)/ len)
    end
    ret(k) = xk
end
ret = ret / len

y = ret
end
\end{lstlisting}

\section{Fast Fourier Transform (DIT-FFT)}

FFT实现分为三个模块:
\begin{enumerate}
\item 主程序\verb|myfft|。本模块封装了整个FFT算法,
包括参数检查,奇偶次序调整和蝶形运算模块。
\item 奇偶次序调整模块\verb|fftmap|。本模块将原始输入序列
次序调整以便于作为蝶形运算模块的输入。
\item 蝶形运算模块\verb|fftreduce|。本模块对调整过次序的
序列进行一系列蝶形运算,并输出FFT结果。
\end{enumerate}
其中,奇偶次序调整模块及蝶形运算均使用了递归算法。
\begin{lstlisting}[caption={myfft.m}]
%!/usr/bin/octave
% FFT implementation
% 2016 Apr 19, Zhou Mo

function X = myfft(x)

%% stage 0, initialize

   [I, J] = size(x);
   X = [];

%% stage 1, argument check

   % I should be 1
   if I ~= 1
      disp('invalid input');
      return;
   end

   % J should be 2^m
   j = J;
   while j > 1
      j = j / 2;
   end
   if j ~= 1
      disp('invalid input');
      return;
   end

%% stage 2, map sequence

   y = fftmap(x);
   y = rot90(eye(length(y))) * y';
   y = y'; % e.g. 04261537

%% stage 3, reduce results

   X = fftreduce(y);
   
   return;
end
\end{lstlisting}

\begin{lstlisting}[caption={fftmap.m}]
% myfft internal module
% mapping util
% note, the output is reversed.

% fftmap(a) =
%    [ fftmap(a'(1:n/2 -1)), fftmap(a'(n/2:n)) ]

function y = fftmap(x)
   if length(x) == 1
      y = x;
   else % length(x) > 1
      % split
      x1 = [];
      x2 = [];
      for i = 1:length(x)
         if mod(i, 2) == 0
            x1 = [ x1, x(i) ];
         else
            x2 = [ x2, x(i) ];
         end
      end
      % recursive
      y = [ fftmap(x1), fftmap(x2) ];
   end
   return;
end
\end{lstlisting}

\begin{lstlisting}[caption={fftreduce.m}]
% myfft internal helper
% fft reduction util

% fftreduce(x) = 
%    [ fftreduce(1st half), fftreduce(2nd half) ]

function y = fftreduce(x)
   if length(x) == 1
      y = x;
   else
      x1 = x(1:((length(x)/2))); % first half
      x2 = x(length(x)/2+1:length(x)); % seconf half
      x1 = fftreduce(x1);
      x2 = fftreduce(x2);

      p = 0:((length(x)/2)-1);
      W = exp(-j * (2*pi)/(length(x)));
      w = W .^ p;

      y = [ x1+x2.*w, x1-x2.*w ];
   end
   return;
end
\end{lstlisting}

\section{Experiment}

\subsection{Experiment Setup}
为测试FFT相对于DFT的性能提升,我们生成一个长度为1024的随机
信号,并分别用本文的DFT实现,本文的FFT实现及Matlab内置的
FFT实现对其进行变换,然后对比它们消耗的时间。
\begin{lstlisting}[caption={main.m}]
% fft benchmark

x = rand(1, 1024);

disp ('dft ...');
tic;
X1 = dft(x);
t0 = toc

disp ('fft ...');
tic;
X2 = myfft(x);
t1 = toc

disp ('fft reference');
tic;
X3 = fft(x);
t2 = toc

disp ('');
disp(sprintf('myfft to dft acceleration ratio %f',
    t0/t1));
disp(sprintf('fft to dft acceleration ratio %f',
    t0/t2));
disp(sprintf('fft is %f times faster than myfft',
    t1/t2));
\end{lstlisting}

\subsection{Result and Analysis}
程序运行结果如下:
\begin{lstlisting}[caption={程序运行结果}]
dft ...
t0 =  22.038
fft ...
t1 =  0.37915
fft reference
t2 =  0.0045090

myfft to dft acceleration ratio 58.125119
fft to dft acceleration ratio 4887.593486
fft is 84.087458 times faster than myfft
\end{lstlisting}
从中可以看出,\verb|myfft|的速度比\verb|dft|快
$58.1$倍,而\verb|fft|比\verb|dft|快$4887.6$倍。

根据理论分析,DFT的计算需要约$2N^2-N$即
$2\times 1024^2-1024 = 2096128$
单位的计算时间,FFT的计算需要约$\frac{3}{2}N \log_2 N$即
$1.5 \times 1024 \log_2 1024 = 15360$单位的计算时间。
也就是说,当DFT的计算时间为$22.038$秒时,理想情况下FFT的计算
时间应该接近
\[ 22.038 \times \frac{15360}{2096128} = 0.16149\]
秒。然而,实际运行消耗时间为$0.38$秒,离估计值还有一定
差距,造成这种情况可能的影响因素包括:
(1) FFT采用了递归算法而非迭代算法。

\section{Conclusion}
FFT相对于DFT有显著的性能提升。实际应用中建议用
编译型语言实现FFT而非解释型语言。

\begin{thebibliography}{9}
\bibitem{bib:dsp} 史林,数字信号处理,科学出版社
\end{thebibliography}

\end{CJK}
\end{document}
