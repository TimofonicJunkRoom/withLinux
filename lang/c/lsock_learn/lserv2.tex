\documentclass[a4paper,12pt]{article}
\usepackage[default]{cantarell}
\usepackage{CJKutf8}
\usepackage{CJKulem}

\begin{document}
\begin{CJK}{UTF8}{gbsn}

\section*{lserv2设计}
	本文档旨在lserv2的设计进行说明。(同时是某课的作业)\newline
	\today \newline
	与本文档相关的源代码文件为\fbox{lserv2.c}以及\fbox{lclie2.c}, 本文档的源文件为\fbox{lserv2.tex},
	lserv2的配置文件目录为 \fbox{config}。 均可从下列地址获得。
	\begin{verbatim}
http://github.com/CDLuminate/lsock_learn/
	\end{verbatim}

\tableofcontents
\newpage

\section{编译和运行}
	这是针对linux编写的Socket程序,在终端下执行
	\begin{verbatim}
$ make
	\end{verbatim}
	即编译好。然后执行
	\begin{verbatim}
$ ./server2
	\end{verbatim}
	即在预定义的端口(2333)上开启了服务器。若要使用服务器,可用Netcat或者telnet这个工具,即
	\begin{verbatim}
$ nc localhost 2333
	\end{verbatim}
	与该服务器进行互动。(lclie2.c也能互动)\newline
	具体可以使用的指令参见下文。
\subsection{配置}
	\begin{description}
	\item[端口] 可用 -p 选项指定端口
	\item[账户] 在 config 目录中进行配置
	\end{description}

\newpage
\section{设计}
	\subsection{实现概要}
		在实现了C/S的相互通信的基础上\footnote{也即,服务器和客户端可以在stdin中写入信息,
		然后发送给对方,对方在stdout中打印这条信息。如此实现互动。},在服务器端添加了功能,
		让服务器不仅能接收并打印收到的信息,还能解析客户端发送的信息。\newline
		lserv2采用的模式是
		\begin{description}
		\item[客户端] 客户端只负责发送和接收信息,剩下的所有处理过程都在服务器端发生。
		用户使用客户端的时候,用户输入指令,指令发送到服务器,服务器解析并进行相应反馈。
		\item[服务器] 接收指令,解析,反馈。具体功能参见下文。
		\end{description}
		这样的设计有一点好处:将客户端的功能缩减到极致(只剩下信息收发功能),于是可以预防比如
		“对客户端进行逆向工程,把身份认证部分填充以\fbox{0x90}\footnote{针对x86架构},于是。。。”这样的情况。亦即预防
		调皮的客户端,甚至不怀好意的重新实现过的客户端。
	\subsection{具体功能}
		以下命令都可以在客户端中手工输入,同时服务器能够反馈:
		\begin{enumerate}
		\item GET: 回传一段预定义的HTML代码
		\item USER: 指令格式为 USER <username>,登记用户名
		\item PASS: 指令格式为 PASS <password>,登记密码
		\item QUIT: 退出程序(客户端和服务器均会退出)
		\item LOGOUT: 登出
		\item SEC: 打印预定义的秘密信息,如果没有成功登录过则会发出警告
		\item \*: 如果遇到其他指令,均不进行额外操作,服务器会提示不支持该指令
		\end{enumerate}

\section{安全分析}
	\subsection{缓冲区溢出}
		lserv2 在实现之初就开始预防缓冲区溢出,具体体现在比如:
		\begin{enumerate}
		\item 避免使用 sprintf, strcpy, strcmp 等函数,一律换成更能预防溢出的函数 snprintf, strncpy, strncmp 等。
		\item instruction 缓冲区有 1024 Byte, 但是进行 read() 调用时最大读取 1023 Byte, 一定程度上能够预防某些调皮的客户端进行溢出测试。
		\end{enumerate}
		因此作者本人也深刻怀疑对该程序进行缓冲区溢出研究的难度,但鉴于没有经验,无法评论。
	\subsection{暴力破解}
		为方便研究,作者使用了固定4位的用户名和密码。因此暴力破解“最多”只需要尝试
		\begin{equation}
Try = 256^{4}
		\end{equation}
		种可能性。实际需要测试的值数量小于上述值。因此暴力破解不失为一种良方。
	\subsection{逆向}
		客户端功能极其精简,逆也逆不出有价值的东西来。毕竟Netcat也能与Server正常通信。
		服务器端逆向不太符合实际情景。
	\subsection{中间人攻击/入侵}
		出于C/S通信采用明文,因此只要获得 C/S/Route 之间任意一方的控制权,进行抓包,
		就能够抓取认证信息。
	\subsection{配置文件}
		用户名和密码均放于配置文件中,文件泄漏会导致风险。

\section{参考以及原型}
	\subsection{参考书籍}
		《UNIX网络编程》
	\subsection{设计模式}
		TCP并发服务器,指令驱动

\section{其他}
	更多细节均在程序注释中写明。
	lserv2\footnote{https://github.com/cdluminate}以及本文档(.tex)遵循MIT许可证。
	\subsection{changelog}
		Date: ?
		\begin{enumerate}
		\item 增加一个读取配置文件(用户名和密码)的部分,避免用户名和密码写死的尴尬。
		\item 增加getopt部分,可以更改监听端口。
		\end{enumerate}
		Date, 2014/11/08
		\begin{enumerate}
		\item 增加select()调用,使lserv2阻塞在select而不是直接阻塞在read()和write()调用。
		\end{enumerate}
		Date, 2014/11/09
		\begin{enumerate}
		\item 增加\fbox{LSOCK\_USE\_IPv4}和\fbox{LSOCK\_USE\_IPv6}两个flag。使得程序可以运行于IPv4或者IPv6上,要切换只需要修改lserv2.c头部的预处理器部分。
		\end{enumerate}

	\begin{verbatim}
Copyright (c) 2014 lumin zhou
	\end{verbatim} 


\end{CJK}
\end{document}
