\documentclass[11pt,a4paper]{article}
% use Cantarell (Gnome) as default English font
%\usepackage[default]{cantarell}
\usepackage{kurier}
% CJK utf8
\usepackage{CJKutf8}
% CJKulem : such as uline
\usepackage{CJKulem}
\usepackage{geometry}
\usepackage{listings}
\usepackage{booktabs}
\usepackage{amsmath} 
\usepackage{amssymb} 
\usepackage{fancyhdr}
\usepackage{indentfirst}
\usepackage{graphicx} 
%\usepackage{natbib}
\usepackage{float}
\usepackage{multicol}
\usepackage{framed}

\usepackage[colorlinks,linkcolor=blue,anchorcolor=blue,citecolor=blue]{hyperref}

\begin{titlepage}
\title{1988 年美赛B: 装货问题}
\author{Lumin}
\date{\today}
\end{titlepage}

% 论文用白色A4纸打印;上下左右各留出至少2.5厘米的页边距
\geometry{left=1cm,right=1cm,top=1cm,bottom=2cm}

% ----start document----
\begin{document}
\begin{CJK}{UTF8}{gkai}
\thispagestyle{empty}
% generate the title page
\maketitle
% this cancels page number display.
% \thispagestyle{empty}

\section{装货问题}
	7种规格箱子装入两节车厢,箱子高度和高度一样不用考虑,只考虑不同的厚度t。
	每一节车厢有 10.2 m 的长度(容纳箱子厚度)容量,载重 40 t。
	对于 C5 C6 C7 限制占用总厚度不超过 302.7 cm。\newline
	\begin{center}
	\begin{tabular}{|c|c|c|c|c|c|c|c|}
	种类 & C1 & C2 & C3 & C4 & C5 & C6 & C7 \\
	\hline
	t/cm & 48.7 & 52.0 & 61.3 & 72.0 & 48.7 & 52.0 & 64.0 \\
	w/kg & 2000 & 3000 & 1000 & 500  & 4000 & 2000 & 1000 \\
	n/件 & 8 & 7 & 9 & 6 & 6 & 4 & 8\\
	\end{tabular}
	\end{center}
	目标:装箱,并使浪费空间最小。

	\subsection{变量定义}
	\begin{multicols}{2}
	\begin{itemize}
	\item $x_{ij}$ 为车厢j上装种类i的数量,整数
	\item $n_i$ 为种类i需要装箱的数量,常数
	\item $w_i$ 为种类i的重量,常数
	\item $t_i$ 为种类i的厚度,常数
	\item $L_j$ 为车厢j的最大装货长度,常数
	\item $W_j$ 为车厢j的最大载重质量,常数
	\item $s$ 为特殊限制,常数
	\end{itemize}
	\end{multicols}

	\subsection{问题模型}
	约束条件如下:
	\begin{itemize}
	\item 两车厢的装箱数量小于等于所有箱子数量
		\[ x_{i1} + x_{i2} \leqslant n_i , ( i = 1,2,\ldots,7) \]
	\item 每一节车厢装货不超过最大长度
		\[ \sum_{i=1}^7 x_{ij} t_i \leq L_j , (j = 1,2 ) \]
	\item 每一节车厢装货不超重
		\[ \sum_{i=1}^7 x_{ij} w_i \leq W_j , (j = 1,2) \]
	\item 对 C5 C6 C7 三类的特殊限制
		\[ \sum_{i=5}^7 t_i x_{i1} + t_i x_{i2} \leq s \]
	\end{itemize}
	优化目标为浪费空间最小,即在约束条件下使用空间最多:
	\begin{equation}
		Z = \sum_{i=1}^7 t_i x_{i1} + t_i x_{i2}
	\end{equation}
	变量限制:\\
		所有变量均大于或等于零,$x_{ij}$为整数。

	\subsection{求解}
	为了方便输入将$x_{21} - x_{27}$ 映射为 $y_1 - y_7$\newline{}
	File M1988.lp :
	\begin{verbatim}
Maximize
 obj: 48.7 x1 + 52.0 x2 + 61.3 x3 + 72.0 x4 + 48.7 x5 + 52.0 x6 + 64.0 x7
      + 48.7 y1 + 52.0 y2 + 61.3 y3 + 72.0 y4 + 48.7 y5 + 52.0 y6 + 64.0 y7
Subject to
 x1 + y1 <= 8
 x2 + y2 <= 7
 x3 + y3 <= 9
 x4 + y4 <= 6
 x5 + y5 <= 6
 x6 + y6 <= 4
 x7 + y7 <= 8
 48.7 x1 + 52.0 x2 + 61.3 x3 + 72.0 x4 + 48.7 x5 + 52.0 x6 + 64.0 x7 <= 1020
 48.7 y1 + 52.0 y2 + 61.3 y3 + 72.0 y4 + 48.7 y5 + 52.0 y6 + 64.0 y7 <= 1020
 2000 x1 + 3000 x2 + 1000 x3 +  500 x4 + 4000 x5 + 2000 x6 + 1000 x7 <= 40000
 2000 y1 + 3000 y2 + 1000 y3 +  500 y4 + 4000 y5 + 2000 y6 + 1000 y7 <= 40000
 48.7 x5 + 52.0 x6 + 64.0 x7 + 48.7 y5 + 52.0 y6 + 64.0 y7 <= 302.7
Bounds
 x1 >= 0
 x2 >= 0
 x3 >= 0
 x4 >= 0
 x5 >= 0
 x6 >= 0
 x7 >= 0
 y1 >= 0
 y2 >= 0
 y3 >= 0
 y4 >= 0
 y5 >= 0
 y6 >= 0
 y7 >= 0
Integers
 x1 x2 x3 x4 x5 x6 x7
 y1 y2 y3 y4 y5 y6 y7
End
	\end{verbatim}
	脚本:
	\begin{verbatim}
read M1988.lp
optimize
disp sol va *
	\end{verbatim}
	运行结果:
	\begin{verbatim}
Incumbent solution
Variable Name           Solution Value
x1                            8.000000
x2                            1.000000
x3                            3.000000
x4                            2.000000
x5                            3.000000
x6                            2.000000
y2                            6.000000
y3                            6.000000
y4                            4.000000
y6                            1.000000
All other variables matching '*' are 0.
	\end{verbatim}

\end{CJK}
\end{document}
% ----end document----
