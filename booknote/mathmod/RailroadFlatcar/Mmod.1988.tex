\documentclass[11pt,a4paper]{article}
% use Cantarell (Gnome) as default English font
%\usepackage[default]{cantarell}
\usepackage{kurier}
% CJK utf8
\usepackage{CJKutf8}
% CJKulem : such as uline
\usepackage{CJKulem}
\usepackage{geometry}
\usepackage{listings}
\usepackage{booktabs}
\usepackage{amsmath} 
\usepackage{amssymb} 
\usepackage{fancyhdr}
\usepackage{indentfirst}
\usepackage{graphicx} 
%\usepackage{natbib}
\usepackage{float}
\usepackage{multicol}
\usepackage{framed}

\usepackage[colorlinks,linkcolor=blue,anchorcolor=blue,citecolor=blue]{hyperref}

\begin{titlepage}
\title{1988 年美赛B: 装货问题}
\author{Lumin}
\date{\today}
\end{titlepage}

% 论文用白色A4纸打印;上下左右各留出至少2.5厘米的页边距
\geometry{left=1cm,right=1cm,top=1cm,bottom=2cm}

% ----start document----
\begin{document}
\begin{CJK}{UTF8}{gkai}
\thispagestyle{empty}
% generate the title page
\maketitle
% this cancels page number display.
% \thispagestyle{empty}

\section{装货问题}
	7种规格箱子装入两节车厢,箱子高度和高度一样不用考虑,只考虑不同的厚度t。
	每一节车厢有 10.2 m 的长度(容纳箱子厚度)容量,载重 40 t。
	对于 C5 C6 C7 限制占用总厚度不超过 302.7 cm。\newline
	\begin{center}
	\begin{tabular}{|c|c|c|c|c|c|c|c|}
	种类 & C1 & C2 & C3 & C4 & C5 & C6 & C7 \\
	\hline
	t/cm & 48.7 & 52.0 & 61.3 & 72.0 & 48.7 & 52.0 & 64.0 \\
	w/kg & 2000 & 3000 & 1000 & 500  & 4000 & 2000 & 1000 \\
	n/件 & 8 & 7 & 9 & 6 & 6 & 4 & 8\\
	\end{tabular}
	\end{center}
	目标:装箱,并使浪费空间最小。

	\subsection{变量定义}
	\begin{multicols}{2}
	\begin{itemize}
	\item $x_{ij}$ 为车厢j上装种类i的数量,整数
	\item $n_i$ 为种类i需要装箱的数量,常数
	\item $w_i$ 为种类i的重量,常数
	\item $t_i$ 为种类i的厚度,常数
	\item $L_j$ 为车厢j的最大装货长度,常数
	\item $W_j$ 为车厢j的最大载重质量,常数
	\item $s$ 为特殊限制,常数
	\end{itemize}
	\end{multicols}

	\subsection{问题模型}
	约束条件如下:
	\begin{itemize}
	\item 两车厢的装箱数量小于等于所有箱子数量
		\[ x_{i1} + x_{i2} \leqslant n_i , ( i = 1,2,\ldots,7) \]
	\item 每一节车厢装货不超过最大长度
		\[ \sum_{i=1}^7 x_{ij} t_i \leq L_j , (j = 1,2 ) \]
	\item 每一节车厢装货不超重
		\[ \sum_{i=1}^7 x_{ij} w_i \leq W_j , (j = 1,2) \]
	\item 对 C5 C6 C7 三类的特殊限制\footnote{在原文(http://www.comap.com/undergraduate/contests/matrix/PDF/1988/1988B.pdf)中
		该条件是这样描述的:" The total space (thickness) occupied by these crates must not exceed 302.7 cm"。我们理解为
		所有车厢上的567类总和不能超过规定数值。}
		\[ \sum_{i=5}^7 t_i x_{i1} + t_i x_{i2} \leq s \]
	\end{itemize}
	优化目标为浪费空间最小,即在约束条件下使用空间最多:
	\begin{equation}
		Z = \sum_{i=1}^7 t_i x_{i1} + t_i x_{i2}
	\end{equation}
	变量限制:\\
		所有变量均大于或等于零,$x_{ij}$为整数。

	\subsection{使用 ILOG CPLEX 程序求解}
	为了方便输入将$x_{11} - x_{17}$ 重定义为 $x_1 - x_7$,$x_{21} - x_{27}$ 重定义为 $y_1 - y_7$\newline{}
	File M1988.lp :
	\begin{verbatim}
Maximize
obj: 48.7 x1 + 52.0 x2 + 61.3 x3 + 72.0 x4 + 48.7 x5 + 52.0 x6 + 64.0 x7
+ 48.7 y1 + 52.0 y2 + 61.3 y3 + 72.0 y4 + 48.7 y5 + 52.0 y6 + 64.0 y7
Subject to
x1 + y1 <= 8
x2 + y2 <= 7
x3 + y3 <= 9
x4 + y4 <= 6
x5 + y5 <= 6
x6 + y6 <= 4
x7 + y7 <= 8
48.7 x1 + 52.0 x2 + 61.3 x3 + 72.0 x4 + 48.7 x5 + 52.0 x6 + 64.0 x7 <= 1020
48.7 y1 + 52.0 y2 + 61.3 y3 + 72.0 y4 + 48.7 y5 + 52.0 y6 + 64.0 y7 <= 1020
2000 x1 + 3000 x2 + 1000 x3 +  500 x4 + 4000 x5 + 2000 x6 + 1000 x7 <= 40000
2000 y1 + 3000 y2 + 1000 y3 +  500 y4 + 4000 y5 + 2000 y6 + 1000 y7 <= 40000
48.7 x5 + 52.0 x6 + 64.0 x7 + 48.7 y5 + 52.0 y6 + 64.0 y7 <= 302.7
Bounds
0 <= x1 <= 8
0 <= x2 <= 7
0 <= x3 <= 9
0 <= x4 <= 6
0 <= x5 <= 6
0 <= x6 <= 4
0 <= x7 <= 8
0 <= y1 <= 8
0 <= y2 <= 7
0 <= y3 <= 9
0 <= y4 <= 6
0 <= y5 <= 6
0 <= y6 <= 4
0 <= y7 <= 8
Integers
x1 x2 x3 x4 x5 x6 x7
y1 y2 y3 y4 y5 y6 y7
End
	\end{verbatim}
	脚本:
	\begin{verbatim}
set mip pool absgap 0.5
set mip pool intensity 4
set mip limits populate 9999999999999999
read M1988.lp
mipopt
populate
disp sol list *
disp sol mem	1	Va *
disp sol mem	2	Va *
disp sol mem	3	Va *
disp sol mem	4	Va *
disp sol mem	5	Va *
disp sol mem	6	Va *
disp sol mem	7	Va *
disp sol mem	8	Va *
disp sol mem	9	Va *
disp sol mem	10	Va *
disp sol mem	11	Va *
disp sol mem	12	Va *
disp sol mem	13	Va *
disp sol mem	14	Va *
disp sol mem	15	Va *
disp sol mem	16	Va *
disp sol mem	17	Va *
disp sol mem	18	Va *
disp sol mem	19	Va *
disp sol mem	20	Va *
disp sol mem	21	Va *
disp sol mem	22	Va *
disp sol mem	23	Va *
disp sol mem	24	Va *
disp sol mem	25	Va *
disp sol mem	26	Va *
disp sol mem	27	Va *
disp sol mem	28	Va *
disp sol mem	29	Va *
disp sol mem	30	Va *
disp sol mem	31	Va *
disp sol mem	32	Va *
disp sol mem	33	Va *
disp sol mem	34	Va *
disp sol mem	35	Va *
disp sol mem	36	Va *
disp sol mem	37	Va *
disp sol mem	38	Va *
disp sol mem	39	Va *
disp sol mem	40	Va *
disp sol mem	41	Va *
disp sol mem	42	Va *
disp sol mem	43	Va *
disp sol mem	44	Va *
disp sol mem	45	Va *
disp sol mem	46	Va *
disp sol mem	47	Va *
disp sol mem	48	Va *
disp sol mem	49	Va *
disp sol mem	50	Va *
disp sol mem	51	Va *
disp sol mem	52	Va *
disp sol mem	53	Va *
disp sol mem	54	Va *
disp sol mem	55	Va *
disp sol mem	56	Va *
disp sol mem	57	Va *
disp sol mem	58	Va *
disp sol mem	59	Va *
disp sol mem	60	Va *
	\end{verbatim}
	运行结果:
\input{M1988.solv}

\end{CJK}
\end{document}
% ----end document----
