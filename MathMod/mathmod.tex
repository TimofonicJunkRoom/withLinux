\documentclass[12pt,a4paper]{article} \usepackage{kurier}
% CJK utf8
\usepackage{CJKutf8}
% CJKulem : such as uline
\usepackage{CJKulem}
\usepackage{amsmath}
\usepackage{amssymb}
\usepackage{multicol}
\usepackage{geometry}
\usepackage[colorlinks,linkcolor=blue]{hyperref}
\usepackage{color}

\columnseprule=0.4pt

\geometry{left=1cm,right=1cm,top=1cm,bottom=2cm}
\begin{titlepage}
\title{Mathematic Modeling the Outline\\ -- 数学建模知识点大纲 --}
\author{$e^{\imath \theta} = \cos(\theta) + \imath \sin(\theta)$ \\ Lumin\\cdluminate@163.com\\\texttt{This Document is distributed under the MIT license.} }
\date{\today}
\end{titlepage}

% ----start document----
\begin{document}
\begin{CJK}{UTF8}{gkai}

\maketitle
\tableofcontents
\newpage

\section{Optimization Model}
\begin{multicols}{2}
	\subsection{Lineaer Programming}
		\subsubsection{Standard Form}
			The object is
			\[ Minimize\, Z = C^T X \]
			Subject to
			\[ Ax = b \]
			Bounds
			\[ x \geqslant 0 \]
			Where
			\[ C = \begin{bmatrix} c_1 \\ c_2 \\ \ldots \\ c_n \end{bmatrix} , x = \begin{bmatrix} x_1 \\ x_2 \\ \ldots \\ x_n \end{bmatrix} , A = \begin{bmatrix} a_{11} & \ldots \\ \ldots  & a_{nn} \end{bmatrix} , b = \begin{bmatrix} b_1 \\ b_2 \\ \ldots \\ b_n \end{bmatrix} \]
		\subsubsection{Standarlization Methods}
			\begin{itemize}
			\item 最大最小转换:$Z' = -Z$
			\item 不等式约束变为等式约束:不等式中添加松弛变量/剩余变量$x_{n+i}$。
			\item $b_i \geq 0, (i = 1,2,\ldots,n)$,乘$-1$反转公式以标准化。
			\item $X_i$ 无限制时,引入非负变量$m,n$,令$x_i = m - n$ 并代入目标方程,转化为非负限制。
			\end{itemize}
		\subsubsection{Solving}
			图解法。单纯形法(需求标准式或矩阵式问题)。
		\subsubsection{Application}
			各类资源规划,投资,生产/存储控制,下料。
	\subsection{Integer Linear Programming}
		分支界定法 Branch\&Cut
	\subsection{Mixed Integer Programming}
	\subsection{Dynamic Programming}
		\subsubsection{Basic Concept}
			\begin{enumerate}
			\item Step $k = 1 \ldots n$
			\item Status $S_k$
			\item Decision $u_k(x_k)$
			\item Policy \[ P_{kj} = { u_k(x_k), \ldots, u_j(x_j) } \]
			\item Status Transition Equation \[ x_{k+1} = T_k (x_k, u_k (x_k) ) , k = 1, \ldots, n \]
			\item Objective function \[ V_{k,n} (x_k, p_{k,n}(x_k) ) = \Phi_k (x_k, u_k, V_{k+1,n}) \]
			\item Step Profit \[ V_{k,n}(x_k, p_{k,n}(x_k) = opt_{k\leq j \leq n} V_j(x_j,u_j) \]
			\item Optimal Value Function \[ f_k(x_k) = opt V_{k,n} (x_k, p_{k,n}(x_k)) \]
			\end{enumerate}
		\subsubsection{Optimization Theory}
			$\circ$ Bellman Optimization Theory
		\subsubsection{逆序法,正序法}
		\subsubsection{Applications}
			When put Dynamic Programming into practice, pay attention to 1. definition of steps; 2. policies allowed within each step; 3. status transition equation.
			\begin{itemize}
			\item Shortest path problem
			\item Knapsack Problem
			\item Resource Assignment
			\item \ldots
			\end{itemize}
	\subsection{Non-Linear Programming}
		\subsubsection{Optimal problem without constraint}
			General form \[ Min\, f(x), x = (x_1, x_2, \ldots, x_n)^T \in R^n \]
			Local solution optimized $\to$ 一阶必要  \[ \nabla f(x^*) = 0 \]
			Local solution optimized $\to$ 二阶充分  \[ \nabla f(x^*) = 0 \, and \,\nabla^2 f(x^*)\,is\,Positive\,definite \]
		\subsubsection{Optimal problem with constraints}
			General form \[ Min\, f(x)\, , x \in R \]
			\[ s.t.\,\, c_i(x) = 0, i \in E = \{ 1,2,\ldots,l \} \]
			\[            c_i(x) = 0, i \in I = \{ l+1,l+2,\ldots,l+m \}\]
			Local solution optimized $\to$ 必要
			\[ \exists vector\, \lambda^* \]
			\[ \nabla_x L(x^*,\lambda^*) = \nabla f(x^*) + \sum_{i=1}^{l+m} \lambda_i^* \nabla c_i(x^*) = 0 \]
			\[ c_i (x^*) = 0, i \in E \]
			\[ c_i (x^*) \leq 0, i \in I \]
			\[ \lambda_i^* \geq 0, i \in I \]
			\[ \lambda_i^* c_i (x^*) = 0, i \in I \]
			Where the $L$ is the Lagronge function \[ L(x,\lambda) = f(x) + \sum_{i=1}^{l+m} \lambda_i c_i (x)\]
		\subsubsection{Application}
			订购/存储模型,投资/组合问题\\
			lingo例子\\
			e.g. \[ min f(x) = 100 (x_2 - x_1^2)^2 + (1 - x_1)^2 \]
			\begin{verbatim}
sets:
  var/1..2/: x;
endsets
[OBJ] min = x(1)^2 + x(2);
[C1] x(1)^2 + x(2)^2 <= 9;
[C2] x(1) + x(2) <= 1;
@for(var: @free(x));
			\end{verbatim}
	\subsection{Software/Tools}
		Matlab/Octave, Lingo/Lindo, IBM ILOG CPLEX.
\end{multicols}

\newpage
\section{Dynamics Model}
\begin{multicols}{2}
	\subsection{Differential Function Model}
		\subsubsection{Exponential Model}
			\[ x(t + \Delta t) - x(t) = k x(t) \Delta t \]
			when $\Delta x \to 0$
			\[ \left\{  \begin{array}{l}
				\frac{dx}{dt} = kx \\
				x(0) = x_0
				\end{array} \right.
			\]
			the solution
			\[ x(t) = x_0 e^{kt} \]
		\subsubsection{SI Model: susceptible infective : Logistic}
			\[ s(t) + i(t) = N \]
			\[ k(s) = \frac{ks}{N} = k(1- \frac{i}{n} )\]
			\[ \left\{ \begin{array}{l}
				\frac{di}{dt} = k(1-\frac{i}{n})s \\
				x(0) = x_0
				\end{array} \right.
			\]
			the solution
			\[ x(t) = \frac{n}{ 1+(\frac{n}{x_0}) e^{-kt} -1   }  \]
		\subsubsection{SIS Model}
			\[ \left\{ \begin{array}{l}
				\frac{dx}{dt} = k(1-\frac{x}{n})x -lx \\
				x(0) = x_0
				\end{array} \right.
			\]
		\subsubsection{SIR Model}
			\[ \left\{ \begin{array}{l}
				\frac{dx}{dt} = \frac{ksx}{n} - lx \\
				\frac{ds}{dt} = - \frac{ksx}{n} \\
				\frac{dr}{dt} = lx \\
				x(0) = x_0,\, r(0) = r_0,\, s(0) = s_0
				\end{array} \right.
			\]
	\subsection{Stability}
		\begin{quote}定理: 设$x_0$ 是微分方程 $ dx(t)/dt = f(x) $ 的平衡点且 $f'(x_0) \neq 0$,
		若$f'(x_0) \le 0 $ 则$ x_0$ 稳定;若 $f'(x_0) \ge 0$,则 $x_0$ 不稳定。
		\end{quote}
	\subsection{Application}
		Best fishing strategy.
	\subsection{Software}
		Matlab:\\
			dsolve, solver = \{ ode45, ode... \}

\end{multicols}

\newpage
\section{Possibility}
\begin{multicols}{2}
	\subsection{Probability Base}
		Include \cite{probbase}
	\subsection{Computer Emulating - Monte Carlo}
		Statistical simulation method: Combine the emulation of
		random events with the probability feature of different kinds of random events.\\
		e.g. Solve $ I = \int_a^b g(x) dx $ with Monte Carlo\\
		sol. Cast uniformly randomized points to the rectangle range, then count how many point
		dropped under the curve $g(x)$.\\
		1. Build the probability model; 2. extract samples from known probability distributions;
		3. Setup statistic variables in need.
		\subsubsection{Monte Carlo Precision}
			CLT: Central-Limit Theorem\\
			1. Randomly Casting Points
				\[ E(X) \approx \bar{p} = \frac{k}{n} \]
			2. Mean Value
				\[ \bar{x} = \frac{1}{n}(x_1 + x_2 + \ldots + x_n) = \frac{1}{n} \sum_{i=1}^n x_i\]
	\subsection{Random Number Generator}
		1. $ x_i = \lambda x_{i-1} (mod M) $.\\
		2. $ x_i = (\lambda x_i + c)(mod M) $.\\
		Tests:
		\begin{enumerate}
		\item Parameter test
		\item Uniformity test
		\item Independency test
		\end{enumerate}
		In matlab:\\
			rand*, normrand
	\subsubsection{Probability - Reference}
		\begin{thebibliography}{9}
		\bibitem{probbase} 概率论与数理统计,浙江大学出版社
		\end{thebibliography}
\end{multicols}

\newpage
\section{Graph Theory}
\begin{multicols}{2}
	\subsection{Basic concept of graph theory}
		\begin{itemize}	
		\item 矩阵\\
			邻接矩阵 node - node\\
			关联矩阵 node - edge\\
			赋权矩阵 weight
		\item 各种类型的图\\
			完全图,立方体(偶图),完全偶图,赋权图,有向图,无向图\\
			子图
				\[ V' \subset V, E' \subset E \]
			生成子图
				\[ V' = V, E' \subset E \]
			补图
		\item 树\\
			树,生成树,无向生成树
		\end{itemize}
	\subsection{一些定理}
		\begin{enumerate}
		\item Given simple graph G contains no isolated node,
			and contains m edges, then the number of all generated
			subgraph is $2^m$
			\[ (_0^m) + (_1^m) + \ldots + (_m^m) = 2^m \]
		\item For $\forall$ undirected graph
			\[ \sum_{v \in V} d(v) = 2m \]
		\item For $\forall$ directed graph
			\[ \sum_{v \in V} d^+(v) + \sum_{v \in V} d^-(v) = m \]
		\end{enumerate}
	\subsection{Shortest Path Problem}
		\subsubsection{Dijkstra Algorithm}
			Base of this algorithm
				\[ d(u_0, \bar{S}) = min_{u\in S, v\in \bar{S}} \{ d(u_0, u) + w (u, v) \} \]
			where $u,v$ denotes the source and destination node,
			$d(u,v)$ denotes the distance between u and v, 
			$S \in V$ and $u_o \in S$ and $\bar{S} = V\\S$.\\
			This algorithm can not only find the shortest path from $u_o$ to $v_0$, but also
			all the shortest path from $u_o$ to any other nodes in graph $G$.
		\subsubsection{Floyd Algorithm}
			Get the minimum distance between to given nodes.
	\subsection{Application}
		\begin{enumerate}
		\item 运输问题
		\item 转运问题
		\item 最优指派问题: 匈牙利算法
		\item 中国邮递员问题: Fleury Algorithm
		\end{enumerate}
	\subsection{Euler Graph and Hamilton Cycle}
		\subsubsection{Traveling Salesman Problem}
			Figure out the Hamilton cycle which possess the minimum weight.\\
			Assume that:
			\begin{itemize}
			\item $w_{ij}$ denotes the distance between city $i$ to city $j$
			\item $x_{ij}$ denotes the decision if going from city $i$ 
			\end{itemize}
			The solution is:
			\begin{equation} Minimize C = \sum_{i=1}^n \sum_{j=1}^n w_{ij} x_{ij} \end{equation}
			s.t. :
			\[ \sum_{i=1}^n x_{ij} = 1, j = 1,2, \ldots, n \]
			\[ \sum_{j=1}^n x_{ij} = 1, j = 1,2, \ldots, n \]
			\[ u_i - u_{j} + n x_{ij} \leqslant n - 1; i \ne j; i,j = 2,3, \ldots, n \]
			\[ x_{ij} = 0 OR 1; i \ne j; i,j = 1,2, \ldots, n \]
			\[ u_{j} \geqslant 0; j = 1,2, \ldots, n \]
			Note that, there are some TSP example program available in the LINGO's user manual\cite{LINGO.pdf}.
			\begin{verbatim}
MODEL:
	! traveling seller;
SETS:
	city/Pe T Pa M N L/: u;
	link(city, city): w,x;
endsets
data:
	!to Pe  T Pa  M  N  L;
	w =  0 13 51 77 68 50
	13  0 60 70 67 59
	51 60  0 57 36  2
	77 70 57  0 20 55
	68 67 36 20  0 34
	50 59  2 55 34  0;
enddata

n = @size(city);
min = @sum(link: w * x);
@for(city(k):
	@sum(city(i)|i #ne# k: x(i,k)) = 1;
	@sum(city(j)|j #ne# k: x(k,j)) = 1;
);

@for(link(i,j)|i #gt# 1 #and# j #gt# 1 #and# 1 #ne# j:
	u(i) - u(j) + n*x(i,j) <= n-1;
);

@for(link: @bin(x));
end
\end{verbatim}
	\subsection{Trees and spinning trees}
		\subsubsection{无向生成树}
			避圈法
		\subsubsection{最优连线问题}
			Kruskal
		\subsubsection{最大流问题}
	\subsection{Python3}
		python3-networkx
	\subsection{Matlab}
		graph* function set:\\
			graphminspantree\\
			graphshortpath\\
\end{multicols}

\newpage
\section{Statistics and Regression}
\begin{multicols}{2}
	\subsection{聚类分析}
	拟合,分段\newline
	K-means:原则上需要预先知道类别数量。
	\subsection{回归分析}
		\subsubsection{主要内容}
			\begin{itemize}
			\item 相关关系.数学表达式
			\item 回归方程,回归预测
			\item 估计的标准误差
			\end{itemize}
		\subsubsection{Single Variable Linear Regression}
			\[ Y = a_0 + a_1 X + \varepsilon \] 
			where
			\[ \varepsilon_i \sim N(0,\sigma^2) \]
			Object:
			\[ Q = \sum{(y - y_c)^2} \]
			Get the parameters: Least square Method
		\subsubsection{Multi Variable Linear Regression}
			TODO: svm :: svr ?
			\[ Y = a_0 + a_1 X_1 + a_2 X_2 + \ldots + a_n X_n + \varepsilon \]
		\subsubsection{Multi Variable Non-Linear Regression}
			\[ Y = a_0 + a_1 x + a_2 x^2 + \ldots + a_n x^n \]
		\subsubsection{keys}
			\begin{itemize}
			\item 选择主成分
			\item 相关系数
			\item 置信区间
			\end{itemize}
\end{multicols}
	
\section{Markov Chain}
\begin{multicols}{2}
	\subsection{placeholder}
\end{multicols}

\section{Fuzzy Mathematics}
\begin{multicols}{2}
	\subsection{Fuzzy Subset}
		While the feature function of certain subset can be represented as the mapping
		\[ X_A : U \to \{0,1\} \]
		where $ X_A (x) = 1 $ when $ x \in A $.\\
		秃头悖论。\\
		The fuzzy subset $A \in U$ can be represented as the mapping
		\[ A(x): U \to [0,1] \]
		\subsubsection{$\lambda$ Cut Set}
			\[ A_\lambda = \{ x | A(x) \geqslant \lambda \} \]
		\subsubsection{Fuzzy Relationship}
			While the classical 2 variable relationship can be represented as the mapping
			\[ R : X \times Y \to \{ 0, 1 \} \]
			which is in fact the feature function of subset $R$ of $X \times Y$.\\
			The fuzzy relationship can be represented as the mapping
			\[ R : X \times Y \to [ 0, 1 ] \]
			The fuzzy matrix means \[ R_{m\times n}(i,j) = R(x_i,y_j) \]
		\subsubsection{Fuzzy Relation Synthesis}
			\[ R_1 \circ R_2 = (c_{ij})_{m\times n} \]
			where
			\[ c_{ij} = \vee \{ (a_{ik} \wedge b_{kj} | 1 \leq k \leq s \}\]
		\subsubsection{Fuzzy Clustering}
			\begin{itemize}
			\item Data Standarlizing\\
				1. 平移:标准差变换,极差变换
			\item Fuzzy Likeness Matrix\\
				1. Cosine of included angle\\
				2. Correlation coefficient\\
			\item Distances $r_{ij} = 1 - c d(x_i, x_j)$\\
				1. Hamming Distance\\
				2. Euclidean Distance\\
				3. Chebyshev Distance\\
			\item Fuzzy equivalant matrix
			\end{itemize}
\end{multicols}

\section{排队论}

\newpage
\appendix
\section{Paper composing}
\begin{multicols}{2}
	\begin{enumerate}
	\item Background of problem
	\item Assumptions
	\item Setup the Mathematical Model
	\item Get the solution of model
	\item Model Analyzation
	\item Model Validation
	\end{enumerate}
	\subsection{Reference Structure}
	摘要
	问题重述与分析
	问题假设
	符号说明
	模型建立与求解
	结果分析
	模型检验
	模型推广
	模型评价
	参考文献和附录
\end{multicols}

\newpage
\section{Common References}
\begin{thebibliography}{9}
\bibitem{LINGO.pdf} LINGO 官方文档 \url{http://www.lindo.com/downloads/PDF/LINGO.pdf}
\bibitem{} 数学建模基础(第二版),薛毅,科学出版社
\bibitem{} \href{https://github.com/CDLuminate/mathmod}{本文的Git Repo,以及我写的一些数模相关程序}
\bibitem{octave.pdf} Octave PDF Document \url{http://www.gnu.org/software/octave/octave.pdf}
\end{thebibliography}

\end{CJK}
\end{document}
% ----end document----
